\documentclass[a6paper]{report}
\usepackage[margin=5mm]{geometry}
\title{Netocracy: A Ludic Politics Simulator}
\author{The Netocracy Team}
\makeindex
\begin{document}
\maketitle
\tableofcontents

\chapter{Introduction}

This project is supposed to be a ludic experiment of politics in a multiplayer simulation game.

\chapter{Base Game}

The following features are to be expected in a first release version.

\section{The Politician}

An enterpreuner that runs for election to the City Council. All players are politicians, and they must secure enough votes to win a chair at the council while managing their businesses.

After being elected, a council member must choose between making proposals at the City Council or mingling with important townsfolk. Proposals are good for expanding the town, which can be profitable for the player's own businesses, whereas mingling can raise his/her own popularity (or lower his/her adversaries'). A politician who is currently not in office may still mingle, so the player will always have something to do.

\section{City Map}

\section{Zones and Buildings}

\section{Proposal}

\section{Mingling Actions}

\chapter{Planned Features}

\section{Building and Upgrading Roads}

\section{News and Media}

\section{Banks}

\section{Council President}

\end{document}